\appendix
\section{Model Priors}
Time to max concentration values for patient $j$ are drawn from a log normal distribution

\begin{equation}\label{eq:eq_6}
t_{\mathit{max}, j} \vert \mu_t, \sigma_t \sim \operatorname{LogNormal}(\mu_t, \sigma_t)
\end{equation}

\noindent and $\alpha$ is drawn from a weakly informative beta prior to prevent degenerate cases when $\alpha$  is 0 or 1

\begin{equation}\label{eq:eq_7}
\alpha_j \sim \operatorname{Beta}(2,2)  \>.
\end{equation}

\noindent The rate constants for patient $j$,  $k_{e,j}$ and $k_{a,j}$, are determined from \cref{eq:eq_4,eq:eq_5}. The clearance rate is modelled hierarchically

\begin{equation}\label{eq:eq_8}
\mathit{Cl}_j \vert \mu_{\mathit{Cl}}, \sigma_{\mathit{Cl}}  \sim \operatorname{LogNormal}(\mu_{\mathit{Cl}}, \sigma_{\mathit{Cl}}) \>.
\end{equation}

\noindent Each patient is observed to have a non-zero concentration at time 0.5, so the time delay for each patient is no larger than 0.5 hours.  We place a beta prior on the delay

\begin{equation}\label{eq:eq_9}
\delta_j \vert \phi, \kappa \sim \operatorname{Beta}(\phi / \kappa, (1-\phi) / \kappa)
\end{equation}

\noindent and multiply delta by 0.5 in our model to ensure the maximum delay is 0.5 hours.  Here, $\phi$ is the mean of this beta distribution and $\kappa$ determines the precision of the distribution. Shown in \cref{net} is a Bayes net to exposit model structure at a high level.

\begin{figure}[h!]
	\centering
	\begin{tikzpicture}
	
	\node[latent](phi){$\phi$};
	\node[latent, right=of phi](kappa){$\kappa$};
	\node[latent, right = of kappa](s_t){$\sigma_t$};
	\node[latent, right = of s_t](m_t){$\mu_t$};
	\node[latent, right = of m_t](s_cl){$\sigma_{\mathit{Cl}}$};
	\node[latent, right = of s_cl](m_cl){$\mu_{\mathit{Cl}}$};
	
	\node[latent, below = of kappa](delta){$\delta$};
	\node[latent, right = of delta](alpha){$\alpha$};
	\node[latent, below = of m_t](tmax){$t_{\mathit{max}}$};
	\node[latent, below = of s_cl](cl){$\mathit{Cl}$};
	
	\node[obs, below = of alpha](t){$t$};
	\node[obs, right = of t](y){$y$};
	\node[latent, right = of t, xshift = 3.25cm](sig){$\sigma_y$};
	
	\edge{phi}{delta};
	\edge{kappa}{delta};
	
	\edge{s_t}{tmax};
	\edge{m_t}{tmax};
	\edge{s_cl}{cl};
	\edge{m_cl}{cl};
	
	\edge{delta}{y};
	\edge{alpha}{y};
	\edge{tmax}{y};
	\edge{cl}{y};
	\edge{t}{y};
	\edge{sig}{y};
	
	\plate{t_y_pairs}{(t)(y)}{$i=1\dots 8$};
	\plate{patient_level}{(t_y_pairs)(delta)(alpha)(tmax)(cl)}{$j = 1 \dots 36$};
	\end{tikzpicture}
	
	
	\caption{Graphical description of the data generating process for our model.  The data consist of 36 patients, indexed by $j$.  Each of the $j$ patients are observed a total of 8 times, with each observation index by $i$.  The data are generated by drawing random variables from their appropriate distribution at the top level and then drawing child random variables directly there after.  As an example, $\phi$ and $\kappa$ are drawn, which are then used to draw the $\delta_j$, which are then used to draw each of the 8 concentration values, $y_i$ for each of the $j$ patients.}
	\label{net}
\end{figure}

\subsection*{Priors for Model Hyperparameters}

Estimates of the time to max concentration for apixaban place the population median $t_{\mathit{max}}$ near 3.3 hours post dose  \citep{Byon2019-gf}. Assuming the median and the mean are similar, this provides information for $\mu_t$ and so we use specify 

\begin{equation}\label{eq:eq_10}
 p(\mu_t) = \operatorname{Normal}(\log(3.3), 0.25)
\end{equation}

\noindent The standard deviation of the prior for $\mu_t$ was selected via prior predictive checks in which profiles are drawn and priors are assessed as realistic or not.  We choose to err on the side of caution and inflate the uncertainty in this estimate to account for population differences between the measured patients in the data and the patients used in studies to determine the estimates of $t_{\mathit{max}}$. The population variability of $t_{\mathit{max}}$ was modeled as

\begin{equation}\label{eq:eq_11}
p(\sigma_t) = \operatorname{Gamma}(10,100)
\end{equation}

\noindent Using these priors, we recover similar median, min, and max $t_{\mathit{max}}$ values as reported by \cite{Byon2019-gf}. Similarly, we model the population mean and variability for the clearance rate as

\begin{align}
	p(\mu_{\mathit{Cl}}) &= \operatorname{Normal}(\log(3.3), 0.15) \label{eq:eq_12} \\
	p(\sigma_{\mathit{Cl}}) &= \operatorname{Gamma}(15, 100) \label{eq:eq_13}
\end{align}

\noindent so that population estimates of the mean clearance rate are near 3.3 litres per hour with inflated uncertainty to account for possible population differences. We use weakly informative priors for $\phi$ and $\kappa$ which induces an approximately uniform prior on $\delta$.

\begin{align}
	 p(\phi) &= \operatorname{Beta}(20,20) \label{eq:eq_14}\\
	 p(\kappa) &= \operatorname{Beta}(20,20)  \label{eq:eq_15}
\end{align}

The tools used to measure the concentration of apixaban are believed to be within 10\% of the real concentration.  This implies that the observational model is heteroskedastic. We use a log-normal likelihood so that positivity of observed concentrations and heteroskedasticity are respected. We place a lognormal prior on the likelihood’s variability with

\begin{align}
	p(\sigma_y)  &= \operatorname{LogNormal}(\ln(0.1), 0.2) \label{eq:eq_16}\\
	C_{j}(t) \vert \mathit{Cl}_{j}, k_{a,j}, k_{a,j}, \delta_j &\sim \operatorname{LogNormal}(\ln(y(t)), \sigma_y)  \label{eq:eq_17}
\end{align}


\subsection*{Posterior Summarization and Generating New Data}

Once our model was fit on the pharmacokinetic data, the marginal posteriors were summarized to create priors for the new model.  Parameters for these priors were determined by using maximum likelihood on the posterior samples.  The priors for the new model are as follows:

\begin{align}
	\mu_{\mathit{Cl}} &\sim \operatorname{Normal}(1.64, 0.09)  \label{eq:eq_18} \\
	\sigma_{\mathit{Cl}} &\sim \operatorname{LogNormal}(-0.94, 0.11)  \label{eq:eq_19} \\
	\mu_{t} &\sim \operatorname{Normal}(0.97, 0.05)   \label{eq:eq_20} \\
	\sigma_{t} &\sim \operatorname{LogNormal}(-1.40, 0.12)  \label{eq:eq_21} \\
	\alpha_j &\sim \operatorname{Beta}(2,2)  \label{eq:eq_22} \\
	\sigma_y &\sim \operatorname{LogNormal}(-1.76, 0.12)  \label{eq:eq_23}
\end{align}

\noindent Lognormal distributions were used to respect positivity of some parameters.  The posterior predictive distribution of the model fit to the data from \cite{Beaton2018-el} was then used to simulate 100  pseudopatients.  The time delay, $\delta$ was not used to generate these data as $\delta$ does not affect the overall shape of the concentration function, it merely shifts it right.  The model with the priors defined by \crefrange{eq:eq_18}{eq:eq_23} was then refit on the 100  pseudopatients in order to examine differences between HMC and MAP in a “best case” scenario. The pseudopatients were sampled between 0.5 and 12.0 hours after ingestion in increments of 0.5. Draws from the posterior were used to predict latent concentration for each patient at times 0.75 to 11.75 in increments of 0.5.